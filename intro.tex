Accurate information on land cover state and change are essential for the management of climate change, natural resources and therefore human wellbeing\cite{alcock2015accounts,burkhard2009landscapes,koschke2012multi}. Land cover is an essential climate variable\cite{bojinski2014concept,vargo2013importance}. Supervised classification of satellite images is the most cost-effective way of deriving large-scale land cover information.  Historical challenges have been access to adequate satellite images and labelled ground reference samples necessary for image classification and product validation.  However, with increasing satellite missions and freely available satellite data, made available through the ESA Sentinel missions and the USGS Landsat programme, image availability is now rarely a problem and a new challenge is how to realise the potential of this vast and rapidly increasing resource.  Ground reference data remain limiting.  Traditionally, ground data collection has required trained surveyors to visit widespread, often inhospitable and difficult to access locations.  This is expensive and on its own is unrealistic to meet modern needs.  In some circumstances the manual interpretation of coincident, high resolution aerial and satellite images to gather training and validation observations is a pragmatic alternative to field visits, but this is still time consuming and requires skilled interpreters.  Crowd-based data collection tools may have potential\cite{laso2016crowdsourcing,comber2013using}.  A promising technique is semi-supervised classification\cite{maulik2011self,liu2013self}, which starts with a set of labelled samples and a much larger set of unknown samples.  The spectral properties of the known samples are used to predict the thematic membership of the unknown samples and if appropriate these are recruited to augment the training set.  Other researchers have investigated the re-use of historically collected ground data \cite{inglada2017operational,tardy2017fusion} or have used existing land cover maps to select training data \cite{kim2017self}. The UK Land Cover Map for 2015 produced by the UK Centre for Ecology and Hydrology (ref) used an historical sequence of three land cover maps (from 1990, 2000 and 2007) to automatically select land cover locations for classifier training.  The work presented here learns from LCM2015 production and represents a significant step towards fully automatic land cover classification.  The process is straightforward.  Training data for classification are automatically selected (bootstrapped) from the most recent land cover map, so each map is produced from the last.  The very first map in a sequence of this kind will have to be produced by conventional classifier training, but this one-off investment then pays off indefinitely.  Intuitively bootstrap training should work well if the target classes are stable over the map refresh interval and the historical map is accurate.  When these conditions are met it should not matter if a small proportion of observations have switched class over time, since the major signal for classification will come from the stable proportion. Conversely, one would expect poor results if the target classes are dynamic over the refresh interval (for example in rotation agriculture) or the initial product has low accuracy.  In this paper we explore the theoretical circumstances in which bootstrap learning is appropriate to help practitioners develop efficient satellite-based land cover monitoring tools.  We perform experiments using simulated satellite data and simulated landscapes with mixed land cover types.  Simulation gives complete, unambiguous knowledge of land cover distribution and therefore eliminates uncertainties associated with real-world landscapes.  It also allows precise manipulation of error-rates in training data and rates of land cover change to fully explore their effects and interactions upon classification accuracy.  Finally to emphasise validity we provide examples of real-landscapes classified using these techniques. 



\paragraph{Usage} Once the package is properly installed, you can use the document class \emph{elsarticle} to create a manuscript. Please make sure that your manuscript follows the guidelines in the Guide for Authors of the relevant journal. It is not necessary to typeset your manuscript in exactly the same way as an article, unless you are submitting to a camera-ready copy (CRC) journal.

\paragraph{Functionality} The Elsevier article class is based on the standard article class and supports almost all of the functionality of that class. In addition, it features commands and options to format the
\begin{itemize}
\item document style
\item baselineskip
\item front matter
\item keywords and MSC codes
\item theorems, definitions and proofs sepurate
\item lables of enumerations
\item citation style and labeling.
\end{itemize}