Accurate information on land cover state and change are essential for the management of natural resources and therefore human wellbeing\cite{alcock2015accounts,burkhard2009landscapes,koschke2012multi}. Land cover is an essential climate variable\cite{bojinski2014concept,vargo2013importance}. Supervised classification of satellite images is the most cost-effective way of deriving land cover information over large areas.  Historical challenges have been the limited availability of affordable, suitable satellite images and labelled ground reference samples for classification and validation.  However, with increasing freely available satellite data, for example from the ESA Sentinel missions and the USGS Landsat programme, image availability is now rarely a problem and a new challenge is how to realise the potential of this expanding resource.  Ground reference data on the other hand remain limiting.  
\paragraph{}Traditionally, ground data collection has required trained surveyors to gather records from widespread, often inhospitable and difficult to access locations.  This is expensive, accounts for the scarcity and is unrealistic to meet modern needs. More innovative approaches are required.  The manual interpretation of coincident, high resolution aerial and satellite images to gather training and validation observations can be a pragmatic alternative to field visits, but this is still time consuming and requires skilled interpreters with detailed landscape knowledge.  Crowd-based data collection tools may have potential\cite{laso2016crowdsourcing,comber2013using,fritz2012geo, fritz2017global} and challenges\cite{comber2016crowdsourcing}.  A promising technique is semi-supervised classification\cite{maulik2011self,liu2013self}, which uses the spectral properties of a set of known samples to predict the thematic membership of the unknown locations and recruits these to the training set.  Other researchers have investigated the re-use of historically collected ground data \cite{inglada2017operational,tardy2017fusion} or have used existing land cover maps to develop training data \cite{kim2017self} (lcm2015 ref?). 
\paragraph{}The UK Land Cover Map for 2015 produced by the UK Centre for Ecology and Hydrology (ref) used an historical sequence of three land cover maps (from 1990, 2000 and 2007) to automatically select stable locations for classifier training.  The work presented here learns from LCM2015 production and represents a significant step towards fully automatic land cover classification and realising the potential of new image resources.  The process is straightforward.  Training data for classification are automatically selected (bootstrapped) from the most recent land cover map, enabling each map to be produced from the last.  Bootstrap training should work well if the target classes are stable over the map refresh interval and the historical map is accurate.  When these conditions are met only a small proportion of the land surface will have changed class over time so the major signal for classification will come from the stable proportion and this should lead to accurate classification. Conversely, one would expect poor results if the target classes are dynamic over the refresh interval (for example in rotation agriculture) or the initial product has low accuracy.  In both of these cases a high proportion of training observations will be wrong.  

\paragraph{}.   For example, errors in the historic map risk perpetual propagation.  But we will show that this is not the general case and more typically, providing the update interval is short relative target class dynamics, historical errors dissipate.   
\paragraph{}

\paragraph{}
In this paper we explore the theoretical circumstances in which bootstrap learning is appropriate to help practitioners develop efficient satellite-based land cover monitoring tools.  To properly understand the implications of bootstrap learning we perform replicated experiments using simulated satellite data and simulated landscapes with mixed land cover types.  Simulation gives precise control of training error and absolute knowledge of land cover distribution, both of which are uncertain in the real-world. It also allows us to manipulate rates of land cover change.  

\paragraph{Hypotheses}
\begin{enumerate}
	\item Errors in land cover and training data dissipate.  Accuracy is asymptotic.
	\item Training volume improves classification results.
\end{enumerate}
 
